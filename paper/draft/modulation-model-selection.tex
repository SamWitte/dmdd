%\documentclass[aps,prd,nofootinbib,amsmath,notitlepage]{revtex4-1}
\documentclass[11pt]{article}
\usepackage{jcappub}
\usepackage{amssymb,esvect,amsmath,graphicx,latexsym,amsthm,slashed,eso-pic,fullpage}
\usepackage[final]{pdfpages}
\usepackage{array}
\usepackage{soul}
\usepackage[normalem]{ulem}
\usepackage{multirow}
\usepackage{pbox}
\DeclareMathOperator{\ev}{eV} \DeclareMathOperator{\kev}{KeV} \DeclareMathOperator{\mev}{MeV} \DeclareMathOperator{\gev}{GeV} \DeclareMathOperator{\tev}{TeV} \DeclareMathOperator{\cm}{cm} \DeclareMathOperator{\barn}{barn} \DeclareMathOperator{\g}{g} \DeclareMathOperator{\km}{km} \DeclareMathOperator{\pb}{pb} \DeclareMathOperator{\s}{s} \DeclareMathOperator{\yr}{yr}\DeclareMathOperator{\gyr}{Gyr} \DeclareMathOperator{\kg}{kg} \DeclareMathOperator{\mpc}{Mpc} \DeclareMathOperator{\few}{few} \DeclareMathOperator{\kel}{K}
\newcommand{\cA}{{\cal A}} \newcommand{\cC}{{\cal C}} \newcommand{\cD}{{\cal D}} \newcommand{\cE}{{\cal E}} \newcommand{\cF}{{ \cal F}} \newcommand{\cH}{{\cal H}} \newcommand{\cJ}{{\cal J}} \newcommand{\cL}{{\cal L}} \newcommand{\cM}{{\cal M}} \newcommand{\cN}{{\cal N}} \newcommand{\cO}{{\cal O}} \newcommand{\cP}{{ \cal P}} \newcommand{\tp}{{ \tilde p}} \newcommand{\cR}{{\cal R}} \newcommand{\cS}{{\cal S}}
\newcommand{\ep}{\epsilon} \newcommand{\vp}{\varphi} \newcommand{\half}{\frac12}
\newcommand{\ie}{{\it i.e.~}}  \newcommand{\eg}{{\it e.g.~}}
\newcommand{\pt}{\partial} \def\d{{\rm d}}  \def\oL{\overline} \def\wh{\widehat}
\newcommand{\pL}{\left(} \newcommand{\pR}{\right)} \newcommand{\bL}{\left[} \newcommand{\bR}{\right]} \newcommand{\cbL}{\left\{} \newcommand{\cbR}{\right\}} \newcommand{\mL}{\left|} \newcommand{\mR}{\right|} \newcommand{\ER}{E_R}
\newcommand{\beq}{\begin{equation}} \newcommand{\eeq}{\end{equation}}
\newcommand{\bea}{\begin{eqnarray}} \newcommand{\eea}{\end{eqnarray}}
\newcommand{\alg}[1]{\begin{align} \begin{split} #1 \end{split}  \end{align}}
\newcommand{\vev}[1]{\langle {#1} \rangle}
\newcommand{\tenx}[1]{\times 10^{#1}}
\newcommand{\Eq}[1]{Eq.~(\ref{#1})} \newcommand{\Eqs}[2]{Eqs.~(\ref{#1}) and (\ref{#2})} \newcommand{\Eqm}[2]{Eqs.~(\ref{#1}) through (\ref{#2})}
\newcommand{\Sec}[1]{Sec.~\ref{#1}} \newcommand{\Secs}[2]{Secs.~\ref{#1} and \ref{#2}} \newcommand{\Secm}[2]{Secs.~\ref{#1} through \ref{#2}}
\newcommand{\Fig}[1]{Fig.~\ref{#1}} \newcommand{\Figs}[2]{Figs.~\ref{#1} and \ref{#2}}
\newcommand{\Tab}[1]{Tab.~\ref{#1}}
\newcommand{\App}[1]{App.~\ref{#1}}
\DeclareMathOperator{\br}{Br} \DeclareMathOperator{\tr}{Tr}
\newcommand{\arxiv}[1]{{\href{http://arxiv.org/abs/#1}{\tt #1}}}

\newcommand{\sdm}[1]{\textcolor{blue}{\textbf{(#1 -- SDM)}}}

\newcommand{\sjwColor}{red}
\newcommand{\sjw}[1]{{\color{\sjwColor} #1}}
\newcommand{\sjwrm}[1]{{\color{\sjwColor}\protect\sout{#1}}}
\newcommand{\sjwrp}[2]{\sjwrm{#1} \sjw{#2}}
\newcommand{\sjwtt}[1]{{\color{\sjwColor}\tt #1}} 


\newcommand{\vgColor}{magenta}
\newcommand{\vg}[1]{{\color{\vgColor} #1}}

\begin{document}
\title{Distinguishing Dark Matter Models With Annual Modulation Signal in Direct Detection Experiments}
\author[a,b]{Samuel J.~Witte,}
\author[c]{Vera Gluscevic,}
\author[d]{and Samuel D.~McDermott}


\affiliation[a]{University of California, Los Angeles, Department of Physics and Astronomy, Los Angeles, CA 90095}

\affiliation[b]{Fermi National Accelerator Laboratory, Center for Particle
Astrophysics, Batavia, IL 60510}

\affiliation[c]{School of Natural Sciences, Institute for Advanced Study, Einstein Drive, Princeton NJ 08540, USA}

\affiliation[d]{C.~N.~Yang Institute for Theoretical Physics, Stony Brook, NY, USA}\subheader{\rm YITP-SB-16-NN}

\abstract{


}

\maketitle

\section{Introduction} \setcounter{page}{2}

A vast array of independent astrophysical and cosmological observations testifyes to the existence of a non-baryonic form of matter in the universe. This so-called dark matter (DM) is the dominant source of the gravitational potential wells that dictate the dynamics and structure of the universe, but dark matter particles have yet to be observed in a laboratory setting. Despite an experimental direct detection program that has now been mature for several decades, all evidence in favor of the existence of dark matter still comes from its gravitational interactions with baryons \cite{Bauer:2013ihz}.

As the next--generation direct detection experiments that incorporate increasingly sensitive detection technologies come online, they will start to probe the final portions of DM parameter space before encountering the irreducible neutrino backgrounds \sjwtt{Add citations. Projections for future DD.} \cite{Cushman:2013zza,Billard:2013qya,Ruppin:2014bra,Davis:2014ama,Dent:2016iht}. Generation 2 (G2) experiments that are currently taking data \cite{} may well be on the cusp of important discoveries, as many interesting theories of DM predict scattering cross sections that live in these portions of paramters space. For example, heavy $SU(2)$--doublet and --triplet fermions, such as the Higgsinos and the wino of supersymmetry, are expected to have cross sections of order $\sigma_{\rm SI} \sim \cO(\few\tenx{-48})\cm^2$ (just about an order of magnitude below the current limits \cite{}), fixed by their Standard Model gauge quantum numbers alone \cite{Hill:2011be,Hill:2013hoa,Hill:2014yxa}, while a heavy $SU(2)$-singlet fermion, like the bino, is around an order of magnitude lower depending on its coannihilation partner \cite{Berlin:2015njh}. Models with kinematically suppressed tree--level scattering may also be embedded in more complete dark sectors that have loop--level cross sections in this same range \cite{Ipek:2014gua,McDermott:2014rqa,Appelquist:2015yfa,Appelquist:2015zfa}.

Because so many theories can be accommodated in the parameter space that will be imminently probed by a variety of experiments, it is timely to plan for the science opportunities associated with the first detection of DM particles. Most notably, in case of a confirmed detection, understanding the high--energy dark sector dynamics will solely rely on examining low--energy recoils of detector elements and solving the ``inverse problem'' to identify the right underlying description of DM--baryon interactions from the experimental measurements. On the other hand, all the information about the dark sector interactions accessible to these measurements is contained within the coefficients of the ``effective field theory of dark matter direct detection'' \cite{Fitzpatrick:2012ix, Anand:2013yka}. This effective description also captures the nontrivial nuclear physics induced by some of the best--motivated UV--complete theories of dark matter \cite{Gresham:2014vja, Gluscevic:2015sqa} through an exhaustable list of nuclear responses. It thus provides a systematic framework for calssifying and describing a wide variety of DM interaction theories, and we will utilize it in this work. 
 
However, due to Poisson noise and degeneracies in the shape of the recoil spectra amongst different interactions, DM model selection, and distinguishing amongst different effective descriptions has been shown to present a difficult task in practice, for a single target material. Recent studies have shown that discriminating between interactions is possible only for strong signals (with hundreds of observed recoil events), and only when measurements on targets with sufficiently diverse nuclear physics characteristics are jointly analyzed \cite{}. Thus, using energy spectra to break degeneracies in the dark matter modeling space crucially relies on ``complementary'' nuclear physics characteristics of available target materials \cite{McDermott:2011hx,Peter:2013aha,Gluscevic:2014vga,Catena:2014epa,Catena:2014hla,Dent:2015zpa,Gluscevic:2015sqa,Ruppin:2014bra}, but even so does not guarantee successful model selection \cite{Gluscevic:2014vga}.

On the other hand, almost since the dawn of direct--detection--related DM studies, the motion of the Earth relative to DM bound in the galactic halo has been predicted to provide a distinctive DM signature \cite{Freese:1987wu, Freese:2012xd,Lee:2013xxa,Britto:2014wga,DelNobile:2015nua,Kouvaris:2015xga} through characteristic annual modulation of the expected event rate of nuclear recoils. Recent work \cite{DelNobile:2015tza,DelNobile:2015rmp} pointed out that non--standard interaction cross sections containing a non--factorizable velocity dependence could produce a modulation signal that is unique to each target element. More generally, a non--trivial velocity dependence in the cross section effectively changes the velocity integral that governs the total event rate in any given experiment, and produces a characteristic modulation signal. It may thus be expected that interaction models that differ solely by the power of velocity of their corresponding cross sections may differ by the phase and/or amplityde of the annual modulation signal they display. 

Motivated by this argument, here we propose that analysis of time dependence of scattering events can help discriminate between interaction models whose recoil energy spectra are otherwise degenerate on a single target material. Using the method of \cite{}, we evaluate the ehancement in prospects for accurate model selection when the annual modulation signal is analysed in combination with recoil--energy measurements in the next--generation direct detection experiments. 


%Almost since the dawn of the study of direct detection experiments, the annual modulation of the rate \cite{Freese:1987wu} arising from the motion of Earth relative to dark matter bound in the galactic halo has been appreciated to provide a distinctive dark matter signature, as have its higher harmonics (see e.g. \cite{Freese:2012xd,Lee:2013xxa,Britto:2014wga,DelNobile:2015nua,Kouvaris:2015xga}). \sjw{Earth's velocity with respect to the dark matter `wind' is at a maximum around June $1^{\rm st}$ and a minimum around December $1^{\rm st}$. For a fixed low energy threshold experiment, conventional wisdom would suggest that increasing the relative velocity between Earth and the dark matter `wind' should enhance the number of dark matter particles which can produce an observable recoil above threshold, and thus the of maximum of the scattering rate should occur around June $1^{\rm st}$. While this is true at large energies, standard differential cross sections (\ie those satisfying $d\sigma/d\ER \propto v^{-2}$) actually produce a phase flip, such that at low energies the rate is maximized in December. This feature occurs because the differential cross section is enhanced by low velocity scattering. }

%The aforementioned arguments, however, only hold for conventional virialized velocity distributions and standard differential cross sections. The disturbance of phase space from dark matter substructure \cite{Green:2000ga,Gelmini:2000dm,DelNobile:2015nua} and the gravitational influence of the solar system \cite{Lee:2013wza,DelNobile:2015nua} have been shown to a non-negligible impact on the annual modulation, in particular modifying the phase of the modulation. Additionally, it was recently pointed out in~\cite{DelNobile:2015tza,DelNobile:2015rmp} that differential cross sections with higher order dark matter velocity dependence produce a phase at low energies that is roughly $5$ months out of phase with conventionally considered interactions. Furthermore, \cite{DelNobile:2015tza,DelNobile:2015rmp} emphasized that differential cross sections containing a non-factorizable velocity and target dependence can produce a modulation that is unique to each target element. A proper analysis of direct detection data including the time information of nuclear recoils may therefore have the potential to elucidate unique astrophysical processes and atypical dark matter-nuclei interactions. \sjwtt{Note to self, add more citations to the above section}  

%Since the local distribution of dark matter is at this moment poorly understood, we will focus in this work on quantitatively understanding the extent to which including information on the modulation of the rate can enhance the ability of future direct detection experiments to correctly differentiate dark matter-nuclei interactions with approximately degenerate recoil spectrum but distinct time behavior. Interestingly, we find that the additional discrimination power from timing information will allow some of the most popular detector designs to go beyond detection and significantly improve model identification, even for models with approximately degenerate recoil spectra.  This is possible because, as previously mentioned, additional factors of $v^2$ appearing in the differential cross section alters the phase of the modulation. 






In \Sec{sec:dd} we review the calculation of the direct detection scattering rate. \Sec{sec:eft} introduces the concept of dark matter effective field theory, discusses how momentum and velocity dependent cross sections can result in nonstandard scattering, and reviews dark matter models which have nearly degenerate recoil spectra but distinguishing time dependence. We introduce our analysis procedure in \Sec{sec:procedure} and present our results in \Sec{sec:results}. We conclude in \Sec{sec:conclusion}.


  

\section{Scattering in Direct Detection Experiments}\label{sec:dd}

\subsection{Basics}

The key measurement of most direct detection experiments is the nuclear recoil energy spectrum---the number count of nuclear recoil events per recoil energy $E_R$, per unit time $t$, per unit target mass,
\begin{equation}
\frac{dR}{dE_R dt}(E_R,t) =  \frac{\rho_\chi}{m_T m_\chi} \int\limits_{v_{\mathrm{min}}}^{v_{\mathrm{esc, lab}}}  v f(\mathrm{\mbox{\bf{v}}},t) \frac{d\sigma_T}{dE_R} (E_R,v) d^3v ,
\label{eq:dRdEr_general}
\end{equation}
where $\rho_\chi$ is the local DM density; $m_\chi$ is the DM particle mass; $m_T$ is the mass of the target nucleus $T$; $\mathrm{\mbox{\bf{v}}}$ is DM velocity vector of magnitude $v$ (in the lab frame); $f(\mathrm{\mbox{\bf{v}}}, t)$ is the observed DM velocity distribution; $d\sigma_T/dE_R=m_T \sigma_T /2\mu_T^2 v^2$ is the differential cross section for DM scattering off a nucleus $T$; and $\mu_T\equiv\frac{m_Tm_\chi}{m_T+m_\chi}$ is the reduced mass of the DM particle and the target nucleus. Integration limits are the minimum velocity od a DM particle needed to produce a nuclear recoil of energy $E_R$, which, for elastic scattering reads $v_\mathrm{min} = \sqrt{m_T E_R/2\mu_T^2}$, and the escape velocity from the Galactic halo in the lab frame, $v_{\mathrm{esc, lab}}$.

The differential rate in \Eq{eq:dRdEr_general} is determined by the experimental setup, the DM astrophysical and particle properties, the nuclear properties of the target material, and the DM--nucleus interaction.\footnote{Throughout this paper we use $T$ to denote the nuclear target and $N$ to denote a nucleon, either neutron $n$ or proton $p$} For the purposes of this study, we set the astrophysical parameters to the following values \cite{Bovy:2013raa,Piffl:2013mla}: $\rho_\chi=0.3$ GeV/cm$^3$; $v_{\mathrm{esc}} = 533$ km/sec (in the Galactic frame), and assume that $f(\mathrm{\mbox{\bf{v}}})$ is a Maxwellian distribution in the Galactic frame, with a rms speed of $155$ km/sec and a mean speed equal to the Sun's rotational velocity around the Galactic center, $v_\textrm{lag}=220$ km/sec.

The underlying particle physics interaction model determines the calculation of the recoil rate through the differential scattering cross section ${d\sigma_T}/{dE_R}$ \cite{Gluscevic:2015sqa,Gresham:2014vja}. This quantity has a normalization (in units of cm${}^2$) which is a free parameter of the model. Different interactions display different functional dependences on $E_R$ and $v$, as discussed in detail in Refs.~\cite{Gluscevic:2015sqa,Gresham:2014vja} and summarized below in \S\ref{subsec:momentum_velocity}.

The total rate $R$ of nuclear recoil events (per unit time and unit mass) is given by the integral of the differential rate within the nuclear--recoil energy window $\cE$ of a given experiment\footnote{For simplicity, we assume unit efficiency of detection within the analysis window, and rescale individual experimetal exposures to take this assumption into account when choosing experimental parameters to represent the capabilities of G2 experiments.}, $R(t)=\int_\cE \frac{dR}{dE_R dt} dE_R$. In turn, the total expected number of events $\langle N_\mathrm{tot}\rangle$ for a fiducial target mass $\cM_\textrm{fid}$, in experiment that started observation at a time $t_1$ and ended at a time $t_2$, within energy window $\cE$ is
\beq
\langle N_\mathrm{tot}\rangle =  \cM_\textrm{fid} \int\limits_{t_1}^{t_2} \int\limits_\cE  \frac{dR}{dE_R dt}(E_R,t)\,dE_R \,dt.
\eeq

\subsection{Momentum and Velocity Dependendence}
\label{subsec:momentum_velocity}

Traditional focus on the two standard scattering cases: spin--independent (SI) and spin--dependent (SD) scattering (where the former that involves coherent contributions from the entire nucleus such that the cross section scales quadratically with nucleon number, while the latter scales with the total nuclear spin, typically concentrated in a single nucleon) obscures the richness of possible phenomenologies accessible to direct detection experiments in cases when these standard--case interactions are suppressed for verious reasons \cite{}. Here we summarize the effective field theory that catalogues all possible energy and velocity dependencies of the cross section, and thus delineates the modeling space for interactions probed by direct detection experiments in most general terms. In the \S\ref{sec:models}, we highlight several well--motivated examples of interesting scattering models which we use in this work to examine the extent to which including the time information of nuclear recoils can enhance model identification.

%The target nucleus in direct detection experiments is chosen to have large principal quantum numbers according to which the experiment is categorized as spin--independent (SI) or spin-dependent. Spin-independent scattering involves coherent contributions from the entire nucleus, such that the cross section scales quadratically with nucleon number. This is to be contrasted with the spin-dependent rate which scales with the total nuclear spin, a quantity that is typically concentrated in a single nucleon. Thus, experiments are usually based on elements selected for having large $A$ or large total spin, with the former experiments dubbed ``spin-independent'' and exhibiting reach to smaller absolute values of the overall cross section. \sjwtt{Guys... I'm gonna want to come back and address this paragraph. I understand the gist and I'm not vehemently opposed to having something like this, but the wording as it just doesn't seem correct.} 

%Such simplistic classification obscures the fact that this dichotomy strictly refers to the dependence on nuclear spin, and does not necessarily imply a hierarchy of expected rates. Many responses sensitive to a diversity of nuclear ``charges'' and ``currents'' can be classified as ``spin-independent'' \cite{Fitzpatrick:2012ix}, most of which are suppressed by small kinematic factors. Suppression of spin-independent scattering can lead to novel signatures in direct detection experiments. Consistently counting the pertinent small factors, and thus acquiring intuition for what might be seen in future experiments, requires a proper effective field theory expansion. 


%The function $\frac{d\sigma_T}{dE_R} (E_R,v)$ in \Eq{eq:dRdEr_general} contains the interesting particle physics of the DM--nucleus scattering and determines the energy and time dependence of the observable recoil event rate. 
The effective field theory of DM direct detection \cite{Fitzpatrick:2012ix, Anand:2013yka} is an expansion in two small kinematic variables: $|\vec q|/m_N$, where $\vec q$ is the change in momentum of the DM particle during the scattering, and $|\vec v_\perp|$ is the orthogonal component of the relative velocity of the initial--state particles. For an incoming (outgoing) DM three--momentum $\vec p(\vec p')$, incoming (outgoing) nuclear three--momentum $\vec k$ $(\vec k')$, and a reduced mass $\mu_{\chi N} = m_\chi m_N/(m_\chi +m_N)$, these kinematic factors are given as $\vec q=\vec p'-\vec p=\vec k-\vec k'$, and $\vec v_\perp=\frac{\vec p}{m_\chi}-\frac{\vec k}{m_N}+\frac{\vec q}{2\mu_{\chi N}}$, respectively.
The momentum transfer is directly related to the nuclear recoil energy as $\vec{q}^{\, 2} =2m_TE_R$. 

These expansion parameters are of the same order of magnitude, but they manifest differently in the observables of the scattering events. In particular, responses \vg{(change ``responses'' to something else)} that enter at higher order in $|\vec q|/m_N$ deliver a vanishing event rate at both small and large momentum transfer (or recoil--energy), with a maximum rate at some intermediate recoil energy, producing a ``turnover feature'' in the recoil--energy spectrum (see .... in Fig.~\ref{}, for example). \vg{What about light-mediator models? There is a second type of spectral signature we are not mentioning, like in the case of millicharge...} On the other hand, higher--order terms in $| \vec v_\perp|$ monotonically decrease with energy, producing monotonic recoil--energy spectra (see .... in Fig.~\ref{}, for example), quantitatively alike to the standard SI or SD interaction spectra. 

While interaction models that feature non--standard momentum dependence can, to an extent, be differentiated using complementary target nuclei with incommensurate nuclear charges \cite{Gluscevic:2015sqa}, the latter class of models---those that differ only by the power of velocity dependence---are more degenerate to each other, especially when measurements on a single target element are considered. In the following, we develop an intuition for how this degeneracy might be overcome, using time dependence of scattering.

\subsection{Time Dependendence}
\label{subsec:time}

In \Eq{eq:dRdEr_general}, the differential rate of nuclear recoils depends on time, which comes as a consequence of the Earth's harmonic annual motion about the Sun. This annual modulation of the DM particle ``wind'' is additionally modified by ``gravitational focusing'' of the wind in Sun's potential well \cite{Danby01021957,Griest:1987vc,Sikivie:2002bj,Alenazi:2006wu}. Overall, these produce an energy--dependent modulation of the total expected recoil rate in direct detection experiments. 

To better illustrate the effects of annual modulation as well as energy and momentum dependence of different interaction scenarios, we show in \Fig{fig:diff_rate_comp} how different momentum and velocity dependences alter the energy (left) and time (right) dependence of the SI, anapole, and heavy--mediator electric dipole (ED--heavy) interactions, assuming a $500$ GeV (top) and $20$ GeV (bottom) DM particle. Instead of plotting the differential rate as a function of time in the right panel of \Fig{fig:diff_rate_comp}, we instead plot the residual, defined as the fractional deviation from the time-averaged energy integrated rate (\ie Residual $\equiv (\bar{R} - R(t))/\bar{R}$). The energy spectra for the SI and ED-heavy interactions are visually distinguishable in a way that the SI and anapole interactions are not: it does not take an arbitrarily large exposure to collect a sufficient number of events to distinguish between the SI and ED-heavy hypotheses using the energy spectrum alone, although doing the same with the SI and anapole hypotheses is quite challenging, given realistic Poisson errors \cite{Gluscevic:2015sqa}. However, as a function of time it is clear that the anapole and SI rates have different phases as functions of time. This difference, expected to be a few percent given standard calculations for the modulation power fraction, is not sufficiently powerful on its own to differentiate two models with similar energy spectra. Nonetheless, we will show below that this small effect can be leveraged to supplement the energy spectrum information, in turn allowing for successful model selection with a smaller number of total events than would be required using energy spectrum information alone.




We take a different approach here, starting from the observation that rates with a response proportional to $| \vec v_\perp|^2$ \sjwrm{include a nonstandard velocity integral in \Eq{eq:dRdEr_general}}  \sjw{produce differential cross sections with non-standard dark matter velocity dependencies}\sjwrm{, leading to a novel dependence on laboratory velocity (as well as the astrophysical parameters) in the differential rate }\cite{Fitzpatrick:2010br} \sdm{I think I prefer the previous version of this} \sjwtt{My primary objection to the previous sentence is the fact that the velocity integral in Eq 2.1 is exactly the same (it is not nonstandard), what is nonstandard is the velocity dependence of the diff cross section. What we could do is rewrite Eq. 2.1 in terms of the `standard' velocity intergral $\eta$, and then state that the velocity integral is nonstanrdard}. Because the laboratory velocity is changing with time, the different dependence on this velocity implicitly leads to a nonstandard time dependence for the interactions whose scattering cross sections include a factor of $|\vec v_\perp|^2$. \sjwtt{Not fanatic about this last sentence. Perhaps this is just my perception, but I don't see the lab velocity as the key feature. Yes, without Earth's rotation there is no time dependence, but a different velocity dependence is not guaranteed to produce a different time dependence (eg $d \sigma / d \ER \propto v^0$ and $v^2$). This is a more complicated effect than I think is suggested.} For large enough dark matter mass, this nonstandard time dependence can \sjwrm{provide the} \sjw{produce a nearly} \sjwtt{(not exactly opposite, closer to 5 months)} {\it opposite phase} compared to the standard rate! \sjw{Furthermore, differential cross sections which contain multiple non-negligible terms with different dependences on $|\vec v_\perp|^2$ can produce an annual modulation that is completely unique to each target element~\cite{DelNobile:2015tza,DelNobile:2015rmp}.} \sjwrm{Thus, a new observable can reveal the novelty of the responses with $|\vec v_\perp|$ dependence: we can use phase information from rates with nonstandard velocity integrals to distinguish these from the standard rates.} \sjw{Thus, by exploiting information on the time dependence of nuclear recoils, it may be possible to distinguish effectively field theory operators with different $|\vec v_\perp|$ dependences.  } \sdm{I think this should be edited along the lines of what we said on Skype call today (Th 12/7/16) but am running out of time to write it in. I support reworking it along those lines, though.} \sjwtt{I don't have time to edit this part right now. I added a section in the intro that discusses the time dependence in more detail, maybe we can reuse that. If neither of you have fixed this by Friday, I'll come back to it then.}

We incorporate this effect following the procedure of~\cite{Lee:2013wza}. 


\section{Distinguishing Scattering Models}\label{sec:procedure}

\subsection{Summary of Models}


\begin{figure*}
\centering
\includegraphics[width=0.49\textwidth, trim=0.cm 0.0cm 0.cm 0.0cm,clip=true]{plots/RecoilComparison_500GeV.pdf}
\includegraphics[width=0.49\textwidth, trim=0.cm 0.0cm 0.cm 0.0cm,clip=true]{plots/Xenon_SIvsAnapole_500GeV_Residual_Theory.pdf}
\includegraphics[width=0.49\textwidth, trim=0.cm 0.0cm 0.cm 0.0cm,clip=true]{plots/RecoilComparison_20GeV.pdf}
\includegraphics[width=0.49\textwidth, trim=0.cm 0.0cm 0.cm 0.0cm,clip=true]{plots/Xenon_SIvsAnapole_20GeV_Residual_Theory.pdf}
\caption{\label{fig:diff_rate_comp}
Comparing the energy (left) and time (right) dependence of the SI (blue), anapole (black), and ED-heavy (green) interactions in a xenon target. \emph{Left:} Differential rate evaluated at June $1^{st}$ as a function recoil energy for a $500$ GeV (top) and $20$ GeV (bottom) dark matter particle. \emph{Right:} Residual, defined to be the fractional deviation in the rate as a function of time (\ie $(\bar{R} - R(t))/\bar{R}$), for a $500$ GeV (top) and $20$ GeV (bottom) dark matter particle. Cross sections have been normalized to the current upper limit. }
\end{figure*}

With this physics motivation in mind, we examine a generic model of new physics: a hidden $U(1)'$ that has several charged fermions $X_i$ and a heavy gauge boson $A'_\mu$ with mass $M$ that kinetically mixes with the Standard Model photon. At high energies, the Lagrangian contains
\beq \label{eq:UV-model}
\cL \supset -  m_X \bar X_i X^i + i \bar X_i \slashed D_{ij} X^j  - \frac12 M^2 A'_\mu A'^\mu  - \frac14 F'_{\mu \nu} F'^{\mu \nu} - \frac\epsilon2 F'_{\mu \nu} F^{\mu \nu}.
\eeq
At low energies, the $A'_\mu$ and most $X$ particles are integrated out. We assume a mass hierarchy that results in an electrically neutral fermion $\chi$ as the lightest degree of freedom in the dark sector. Because of the kinetic mixing, the state $\chi$ couples to the Standard Model nucleon current \cite{Gresham:2014vja},
\beq \label{eq:current}
\cJ_\mu = \partial^\alpha F_{\alpha \mu} = e \sum_{n,p} \bar N \pL Q_N \frac{K_\mu}{2m_N} -\widetilde \mu_N \frac{i \sigma_{\mu \nu}q^\nu}{2m_N} \pR N,
\eeq 
where $Q_{p(n)}=1(0)$ are the nucleon charges in units of the electron charge $e$, $K_\mu/2 = (k_\mu + k'_\mu)/2$ is the average nucleon momentum, and $\tilde{\mu}_N = {\text{magnetic moment} \over \text{nuclear magneton}}$ is the dimensionless magnetic moment of the nucleon.

The details of the masses and charges of the dark fermions $X_i$ that constitute or couple to the dark matter $\chi$ will determine the response that is measured in an experiment. We will use $\cO_\chi^\mu$ to denote the Lorentz-vector fermion bilinear that couples to the current in \Eq{eq:current}. Because we assume $\chi$ is electromagnetically neutral, the possible $\cO_\chi^\mu$ are \cite{Gresham:2014vja, Gluscevic:2015sqa}
\begin{eqnarray} \label{eq:photon-DM-ops}
\cO_{\rm \chi, Anapole}^\mu & = & g^{\rm Anapole}\bar \chi \gamma^\mu \gamma_5 \chi, \\
\cO_{\rm \chi, MD}^\mu & = & \frac{g^{\rm MD}}{\Lambda}\bar \chi i \sigma^{\mu \nu} q_\nu \chi ,\\
\cO_{\rm \chi, ED}^\mu & = & \frac{g^{\rm ED}}{\Lambda} \bar \chi i \sigma^{\mu \nu} \gamma_5 q_\nu \chi.
\end{eqnarray}
As stated above, the interaction operator for $\chi$ is determined by the dynamics of the $X$ fermion(s). The anapole current in \Eq{eq:photon-DM-ops} will arise if charged $X^\pm$ states condense to form a neutral Majorana state $\chi$ \cite{Bagnasco:1993st}. The dipole currents form if an electromagnetically neutral $X^0$ couples to an electromagnetically charged pair of partner $X^\pm$ particles (of appropriate spin) \cite{Weiner:2012gm}. The scale at which the charged $X$ states are integrated out is $\Lambda$.

The simplicity of the model in \Eq{eq:UV-model} and the rich assortment of momentum and velocity dependence that appear in the associated EFT responses illustrates how generic the ``novel responses'' are. We list the EFT classification of these operators in \Tab{tab:operators}. (This is an abbreviated version of the more exhaustive table that appeared in \cite{Gluscevic:2015sqa}, using results of \cite{Gresham:2014vja, Gluscevic:2015sqa}.% For additional details of the implementation of the effective theory in the context of model selection, we refer to the discussion in \cite{Gluscevic:2015sqa}.
) In this work, we will focus on differentiating responses that have the same momentum scaling but different velocity dependence.



\begin{table}[tb]
\begin{centering}
\renewcommand{\arraystretch}{1.3}
\begin{tabular}{c |>{$}c<{$}| >{$}c<{$} >{$}c<{$} c } \hline
 Model name & {\rm Lagrangian} & \text{$\vec q$, $v$ Dependence} &  {\rm Response}  
\\ \hline 
 SI & \frac g{M^2}\bar \chi \chi \bar N N & 1 & M
\\ \hline 
 \multirow{2}{*}{Anapole} & \multirow{2}{*}{$\frac g{M^2}\bar{\chi} \gamma^\mu \gamma_5 \chi \, \cJ_\mu $} & v_\perp^2 & M \\  
 & & \vec{q}^2/m_N^2 & \Delta + \Sigma' 
\\ \hline
\multirow{2}{*}{\pbox{20cm}{Magnetic Dipole (Heavy)}} & \multirow{2}{*}{$\frac g{\Lambda M^2} \bar{\chi} \sigma^{\mu \nu} \chi  \, q_\nu \cJ_\mu $} & \frac{\vec q^{\,4}}{\Lambda^4}+ \frac{\vec{q}^2 v_\perp^2 }{\Lambda^2} & M \\
 & & \vec q^{\,4}/\Lambda^4 & \Delta + \Sigma' 
\\ \hline
Electric Dipole (Heavy) & \frac g{\Lambda M^2} \bar{\chi} \sigma^{\mu \nu} \gamma_5 \chi \, q_\nu \cJ_\mu  & \vec{q}^2 /\Lambda^2 & M 
\\ 
\hline 
\multirow{2}{*}{\pbox{20cm}{Magnetic Dipole (Light)}} & \multirow{2}{*}{$\frac g\Lambda \bar{\chi} \sigma^{\mu \nu} \chi F_{\mu\nu} $} & 1+ \frac{v_\perp^2 m_N^2}{\vec{q}^2 } & M \\
  & & 1 & \Delta + \Sigma' 
 \\ \hline
 Electric Dipole (Light) & \frac g\Lambda \bar{\chi} \sigma^{\mu \nu} \gamma_5 \chi F_{\mu\nu}  & m_N^2/\vec{q}^2 & M 
 \\ \hline 
\end{tabular}
\caption{Selection of operators along with their EFT dependences, adapted from \cite{Gluscevic:2015sqa}. The labels `Light' and `Heavy' in the dipole models denote the magnitude of the mediator mass relative to the characteristic momentum transfer. The nucleon electromagnetic current $\cJ_\mu$ is defined in \Eq{eq:current}; the transverse velocity $v_\perp$ and three--momentum transfer $\vec q$ are defined in terms of the collision momenta in \Eq{eq:kinematic-definitions}; and $\Lambda$ is a heavy mass or compositeness scale appearing in the dipole models. The parametric momentum and velocity dependences (third column) schematically multiply the adjacent EFT response (fourth column). }
\label{tab:operators} 
\end{centering}
\end{table}


\begin{table*}[t] 
\setlength{\extrarowheight}{3pt}
\setlength{\tabcolsep}{12pt}
\begin{center}
\begin{tabular}{c||m{3cm}|m{3cm}|m{3cm}}
Interaction /target & Xe & Ge & F\\
\hline\hline 
$m_\chi$ [GeV] & (20, 125, 500) & (20, 125, 500) & (20, 125, 500) \\
\hline\hline 
SI& (103, 99, 98) & (9, 4, 4)& (5, 1, 2)\\ \hline
%Millicharge& (103, 103, 102)& (834, 203, 189)& (334, 107, 103)\\
%SD flavor-univ.& (103, 97, 95)& (11, 4, 4)& (2318, 708, 725)\\
Anapole& (103, 97, 96)& (11, 5, 5)& (36, 3, 3)\\ \hline
Mag. dip. heavy& (103, 89, 87)& (3, 4, 5)& (4, 1, 1)\\ \hline
Mag. dip. light& (103, 101, 101)& (34, 14, 14)& (86, 16, 15)\\ \hline
Elec. dip. heavy& (103, 91, 88)& (4, 4, 4)& (1, 0, 0)\\ \hline
Elec. dip. light& (103, 102, 101)& (61, 15, 14)& (40, 12, 12)\\ \hline \hline
\end{tabular}
\end{center}
\caption{Predicted number of events in Gen2 experiments for various interactions with xenon, germanium, and fluorine targets, and assuming a dark matter mass of ($20$ GeV, $125$ GeV, and $500$ GeV). The predicted number of events are calculated using a cross section set to the current 90\% upper limit. Labels `light' and `heavy' denote the relative relation between the mediator mass and the characteristic scale of momentum transfer. }
\label{tab:pred_events}
\end{table*}

%\section{Model Selection} \label{sec:procedure}
%In the following, we apply Bayesian model selection to assess the extent to which analyzing the time dependence of nuclear recoils can improve the ability of future direct detection experiments to properly identify the true dark matter-nuclei interaction. 

\subsection{Simulations\label{sec:sims}}

\begin{table*}[tbp]
  \setlength{\extrarowheight}{3pt}
  \setlength{\tabcolsep}{10pt}
  \begin{center}
	\begin{tabular}{c|m{2.3cm}m{4.2cm}m{2.8cm}}  
	Label & A (Z) & Energy window [keVnr] & Exposure [kg-yr] \\
	\hline
	Xe & 131 (54) & 5-40 & 2000 \\
	Ge & 73 (32) & 0.3-100 & 100  \\
%	I & 127 (53) & 22.2-600 & 212 \\
	F &  19 (9) & 3-100 & 606 \\
%	Na & 23 (11) & 6.7-200 & 38 \\
%	Ar & 40 (18) & 25-200 & 3000 \\
%	He & 4 (2) & 3-100 & 300 \\

	\hline
	Xe(x3) & 131 (54) & 5-40 & 6000 \\
	Xe(x10) & 131 (54) & 5-40 & 20 000 \\
	XeG3 & 131 (54) & 5-40 & 40 000 \\
%	I+ & 127 (53) & 1-600 & 424 \\
%	F+ &  19 (9) & 3-100 & 1200 \\
	\end{tabular}
  \end{center}
\caption{Mock experiments considered in this work. The efficiency and the fiducialization of the target mass are included in the exposure. The first group of experiments is chosen such to be representative of the reach of Gen2 experiments for Xe, Ge, and F. The exposure for Xe and Ge is chosen to agree with the projected exclusion curves for LZ and SuperCDMS presented in Ref.~\cite{Cushman:2013zza}. The second group of experiments is used to quantitatively assess the impact of including the timing information of nuclear recoils, in the analysis of a single target experiment, as a function of the observed number of events. }
\label{tab:experiments}
\end{table*}



To simulate future datasets, we assume the dark matter-nuclei interaction is given by one of the models in the left column of Table~\ref{tab:pred_events}, and that the dark matter mass is either $20$ GeV, $125$ GeV, or $500$ GeV. We then optimistically assume that direct detection experiments are on the verge of of detecting dark matter, and set the cross section to be the value maximally allowed by LUX~\cite{Akerib:2016vxi}. The predicted number of events in various Gen2 experiments (see Table~\ref{tab:experiments}) using this cross section for each interaction and mass are shown in Table~\ref{tab:pred_events}. 

For each simulation, the observed number of events is obtained by randomly selecting from a Poisson distribution with a mean given by the predicted number of events. The recoil energy and time of each event is then obtained by applying a rejection sampling algorithm to the two-dimensional differential scattering rate. 

We perform our analysis on a variety of futuristic direct detection experiments. Our initial analysis focuses on the potential of Gen2 experiments to differentiate pairs of interactions with highly degenerate recoil spectra (\eg the SI and anapole interaction). Specifically we focus on xenon, germanium, and fluorine based experiments. Since fluorine experiments measure only the energy integrated rate, information on recoil energies of individual events are always neglected.  The exposure and energy window of these experiments are summarized in Table~\ref{tab:experiments}. Throughout our analysis we assume unit detection efficiency and zero background. In addition to the aforementioned, we also consider the potential reach of a Generation 3 (Gen3) xenon experiment, as well as various xenon experiments with exposures lying somewhere between Gen2 and Gen3 (the properties of which are summarized in Table~\ref{tab:experiments}).    

\begin{figure}
\centering
\includegraphics[width=0.45\textwidth]{plots/PDF_Single_500GeV_Anapole_50sims_Xe_vs_FGeXe_GF_TNT.pdf}
\includegraphics[width=0.45\textwidth]{plots/PDF_Single_500GeV_SI_Higgs_50sims_Xe_vs_FGeXe_GF_TNT.pdf}

\includegraphics[width=0.45\textwidth]{plots/PDF_Single_125GeV_Anapole_50sims_Xe_vs_FGeXe_GF_TNT.pdf}
\includegraphics[width=0.45\textwidth]{plots/PDF_Single_125GeV_SI_Higgs_50sims_Xe_vs_FGeXe_GF_TNT.pdf}

\includegraphics[width=0.45\textwidth]{plots/PDF_Single_20GeV_Anapole_50sims_Xe_vs_FGeXe_GF_TNT.pdf}
\includegraphics[width=0.45\textwidth]{plots/PDF_Single_20GeV_SI_Higgs_50sims_Xe_vs_FGeXe_GF_TNT.pdf}

\caption{\label{fig:gen2}
Model selection prospects with complimentary Gen2 targets. The normalized probability distribution functions for the probability of correctly identifying the true model are shown for the anapole (left) and SI (right) interactions, assuming a $500$ GeV (top), $125$ GeV (middle), and $20$ GeV (bottom) dark matter particle. Results are shown for a xenon experiment (red), and a combined analysis of xenon, germanium, and fluorine experiments (purple), including (solid) and neglecting (dashed) information on the modulation of the rate. Success rate is defined to be the fraction of realizations which produce correct model identification at the level of $ \geq 90\%$}
\end{figure}



\subsection{Analysis method}

Within the Bayesian inference framework, the probability that the data $\vec{X}$ assigns to a given model $\cM_j$ is given by
\begin{equation}\label{eq:probs}
P(\cM_j) = \frac{\cE_j(\vec{X}|\cM_j)}{\sum_i \cE_i(\vec{X}|\cM_i)} \, ,
\end{equation}
where $\cE(\vec{X}|\cM)$ is the evidence of model $\cM$, defined by
\begin{equation}\label{eq:evidence}
\cE(\vec{X}|\cM) = \int d\Theta \, \cL(\vec{X}|\Theta,\cM) \, p(\Theta,\cM) \, ,
\end{equation}
and is intuitively understood to be the factor required to normalize the posterior $\cP$, \ie
\begin{equation}\label{eq:posterior}
\cP(\Theta | \vec{X}, \cM) = \frac{\cL(\vec{X}|\Theta,\cM)\, p(\Theta,\cM)}{\cE(\vec{X}|\cM)} \, . 
\end{equation}
Here, $\cL(\vec{X}|\Theta,\cM)$ is the likelihood, \ie the probability of obtaining the data, given a particular model $\cM$ and parameters $\Theta$ (for the purpose of this analysis $\Theta = \cbL m_\chi, \sigma_p \cbR$), and $p(\Theta, \cM)$ is the prior. In order to remain as agnostic as possible, we take wide priors in both $m_\chi$ and $\sigma_p$\footnote{Log priors are taken for both $m_\chi$ and $\sigma_p$, spanning $1-3000$ GeV in mass and $7$ orders of magnitude in cross section.}. In our analysis we use an unbinned extended likelihood function of the form
\begin{equation}\label{eq:likelihood}
\cL(\vec{X}|\Theta,\cM) = \frac{\mu^N}{N!} \, e^{-\mu} \, \prod_{x_i \in \vec{X}}\, \frac{1}{\mu} \, \frac{dR}{d\ER dt} \bigg|_{\ER,t \, = \, x_i} \, ,
\end{equation}
where $\mu$ is the predicted number of events, $N$ is the number of observed events, and the product runs over all the normalized differential rate evaluated at the $\ER$ and $t$ values of each observed event $x_i \equiv \cbL E_{R,i} \, , \, t_i \cbR$. When time or $\ER$ information is neglected, the differential rate is implicitly understood to be averaged over that variable. 

Our analysis proceeds as follows. We begin by simulating data for an experiment (or experiments) assuming a particular dark matter model, mass, and cross section (see Sec.~\ref{sec:sims}). We then use PyMultiNest to reconstruct the posterior defined in \Eq{eq:posterior}, and subsequently calculate the evidence for various dark matter models~\cite{pymultinest,Feroz:2008xx}\footnote{Multinest runs are performed with 2000 live points, an evidence tolerance of 0.1, and a sampling efficiency of 0.3.}. Once the evidence of various models has been computed, one can estimate the probability of successfully being able to identify the true model using \Eq{eq:probs}. This procedure is then repeated for $\simeq \cO(50)$ simulations to assess the variability in successful model identification arising from Poisson fluctuations. A model is said to be correctly identified if the probability determined using \Eq{eq:probs} is large. For the purpose of this paper, we define the boundary for successful model identification at $P \geq 90\%$. The primary quantity of interest for future direct detection experiments is then the fraction of simulations which lead to a successful model identification.    



Instead of plotting the individual probabilities of each simulation, we apply kernel density estimation (KDE) with a Gaussian kernel to determine the distribution functions of these probabilities. For of the results listed in the following section, we plot the KDE distribution for each experimental combination both with and without time, and determine the fractional success rate by integrating the distribution above the 90\% threshold. 


\begin{figure*}
\centering
\includegraphics[width=0.7\textwidth]{plots/PDF_20GeV_Anapole_50sims_Xe_Xe3x_Xe10x_XeG3_GF_TNT.pdf}
\caption{\label{fig:20gev_anapole_XeFull_TNT_GF}
Model selection prospects for a single target (xenon), including (solid) and neglecting (dashed) information on the modulation of the rate. The normalized probability distribution functions are plotted for the probability associated with correcting identifying a 20 GeV dark matter particle scattering through the anapole interaction. Panels from left to right, top to bottom, correspond to exposure of 2 ton-years (blue), 6 ton-years (red), 20 ton-years (green), and 40 ton-years (magenta). Calculations are performed assuming the true cross section is sitting at the current $90\%$ upper limit. Successful model identification, defined as having $\geq 90 \%$ probability, are provided in the legend of each analysis.}
\end{figure*}



\section{Results}\label{sec:results}
We begin by considering the extent to which including time information in the analysis of Gen2 experiments can assist in breaking degeneracy between the SI and anapole interactions. Specifically, in \Fig{fig:gen2} we consider the probability of correctly identifying the anapole (left) and SI (right) interactions, assuming a putative signal is detected in either a xenon experiment (red) or in a combination of xenon, germanium, and fluorine experiments (purple). Analysis is shown for a $500$ GeV dark matter particle (top), $125$ GeV dark matter particle (middle), and a $20$ GeV dark matter particle (bottom), including (solid) and neglecting (dashed) information on the modulation. 

Consistent with the results of~\cite{Gluscevic:2015sqa}, we find that both models can be correctly identified in Gen2 experiments for a $20$ GeV dark matter particle, but only if detections are made in both xenon and fluorine based experiments (note that Xe+Ge analyses do not break the degeneracy and have no model discrimination). Should detections be made in both xenon and fluorine experiments, time information would not be needed to differentiate the two models.  



Correct model identification is slightly more complicated for heavier dark matter candidates due to the reduced scattering rate in fluorine. For the anapole model, detection in Xe leads to a model discrimination of a few percent. If however, detections are made in xenon, germanium, and fluorine experiments, model selection could be as high as $\simeq 20 \%$. Including time information only negligibly increases model selection in both cases. Model selection for the SI interaction is significantly reduced relative to that of the anapole interaction. Xenon experiments are completely unable to break this degeneracy if massive dark matter interacts through a SI interaction, regardless of whether or not time information is included in the analysis. Should detections be made in xenon, germanium, and fluorine experiments, correct model identification is still only at the level of a few percent. As with the anapole interaction, including time information in the analysis does not impact the ability of these experiments to correctly identify the underlying model. 

\begin{figure*}
\centering
\includegraphics[width=0.7\textwidth]{plots/PDF_125GeV_Anapole_50sims_Xe_Xe3x_Xe10x_XeG3_GF_TNT.pdf}
\caption{\label{fig:125gev_anapole_XeFull_TNT_GF}
Same as Fig.~\ref{fig:20gev_anapole_XeFull_TNT_GF} but for $125$ GeV dark matter.}
\end{figure*}

Given that Gen2 experiments will optimistically detect $\simeq 100$ events for the SI and anapole interaction, it is not surprising that analyzing Gen2 data with time has a minimal effect on model selection (see \eg Sec.~4 of \cite{DelNobile:2015nua} for an estimation of the number of events needed for phase discrimination). Perhaps the more interesting question is: how many events are needed before the inclusion of time information can significantly effect model selection? We address this question in the context of the breaking the SI and anapole degeneracy in xenon based experiments in Figs.~\ref{fig:20gev_anapole_XeFull_TNT_GF}-\ref{fig:500gev_anapole_XeFull_TNT_GF}. For 20, 125, and 500 GeV anapole dark matter, we plot the probability of correctly identifying the true model in a 2 ton-year (blue), 6 ton-year (red), 20 ton-year (green), and 40 ton-year (magenta) xenon experiment, neglecting (dashed) and including (solid) time information. Since results for the SI interaction are qualitatively similar to that of the anapole, we defer the SI analogues of Figs.~\ref{fig:20gev_anapole_XeFull_TNT_GF}-\ref{fig:500gev_anapole_XeFull_TNT_GF} to Appendix A. 

\Fig{fig:diff_rate_comp} clearly shows that the recoil spectrum observed in xenon arising from light anapole and SI dark matter is more degenerate than that of heavy anapole and SI dark matter. Unfortunately the phase of light dark matter is also degenerate for conventional velocity dependent cross sections (\ie $d\sigma/d\ER \propto v^{-2}, v^0, v^2, ...$)~\cite{DelNobile:2015tza,DelNobile:2015rmp}, suggesting that the inclusion of time information in the analysis may not be sufficient to fully differentiate these models. Indeed \Fig{fig:20gev_anapole_XeFull_TNT_GF} shows that even in a Gen3 xenon experiment, the SI and anapole interaction can only be correctly identified in $\simeq 15\%$ of our simulations (assuming time is included). Perhaps surprisingly, however, is that the inclusion of time in the analysis improves model discrimination by $\simeq 15\%$. This improvement arises not from differences in the phase, but rather from the amplitude of the modulation, which as shown in \Fig{fig:diff_rate_comp} differs by $\simeq 5\%$.



As the dark matter mass increases the recoil spectrum become less degenerate, allowing for better model discrimination at fixed exposure. Additionally, heavier dark matter allows these experiments to probe regions of parameter space where the anapole and SI interaction are out of phase (see \Fig{fig:diff_rate_comp}). Figs.~\ref{fig:125gev_anapole_XeFull_TNT_GF} and \ref{fig:500gev_anapole_XeFull_TNT_GF} confirm that heavier dark matter candidates allow for better model discrimination of these models, particularly when time is included in the analysis. We emphasize that including time in the analysis can potentially improve model selection in Gen3 experiments by as much as $\simeq 30\%$ for heavy dark matter candidates, where the phase of these interactions may be misaligned by as much as $\simeq 5$ months. Despite this improvement, Gen3 experiments will likely also need to exploit target complimentarily to in order to fully break the SI-anapole degeneracy. 

To briefly summarize the conclusions of Figs.~\ref{fig:20gev_anapole_XeFull_TNT_GF}-\ref{fig:500gev_anapole_XeFull_TNT_GF}, including time information in the analysis of future xenon direct detection experiments begins to significantly enhance model selection after $\simeq \cO(1000)$ ($ \cO(300)$) events have been observed, assuming a 20 (500) GeV dark matter particle. Gen3 xenon experiments can perhaps expect an enhancement in model selection as large as $30\%$ (for the SI this enhancement can be as large as $\simeq 40\%$) for the most optimistic of circumstances, but will still require target complementarity to definitely identify the correct model.  

\begin{figure*}
\centering
\includegraphics[width=0.7\textwidth]{plots/PDF_500GeV_Anapole_50sims_Xe_Xe3x_Xe10x_XeG3_GF_TNT.pdf}
\caption{\label{fig:500gev_anapole_XeFull_TNT_GF}
Same as Fig.~\ref{fig:20gev_anapole_XeFull_TNT_GF} but for $500$ GeV dark matter.}
\end{figure*}

We would like to emphasize that the purpose of this paper is not of a comparison of the SI and anapole interactions, but rather a quantitative assessment of whether time information can be exploited in future direct detection analyses to break degeneracies in the recoil spectrum. The SI and anapole interactions provide one particularly illuminating example of this because they contain different dependencies on the dark matter velocity (and consequently have a different modulation). Despite also having an approximately degenerate recoil spectrum in xenon, the same cannot be said for the SI and SD interactions which have the same dark matter velocity dependence. There do exist, however, other illustrative examples which we briefly consider below.

In \Figs{fig:125gev_Mag.dip.heavy_XeFull_TNT_GF}{fig:125gev_Mag.dip.light_XeFull_TNT_GF} we consider a comparison of the magnetic dipole and electric dipole interactions for a $125$ GeV dark matter particle, assuming a heavy (\Fig{fig:125gev_Mag.dip.heavy_XeFull_TNT_GF}) and light (\Fig{fig:125gev_Mag.dip.light_XeFull_TNT_GF}) mediator. As before we consider putative detections in future xenon experiments with exposures varying from 2 ton-years to 40 ton-years. The results are rather similar to the SI and anapole comparison in that Gen3 experiments can expect a $\simeq 20\%$ improvement in model selection when time is included in the analysis, but again necessitate target complementarity to fully differentiate these models.








\section{Conclusions}\label{sec:conclusion}
We have considered here the potential impact of using time information in the analysis of future direct detection experiments to break approximate degeneracies that may appear between the recoil spectrum of different interactions. Specifically, we have applied Bayesian model selection to simulated data sets that include the impact of Poisson fluctuations to quantitively assess future ability of xenon experiments to successfully identify approximately degenerate models, assuming analyses include and neglect the time information of detected recoils.

 
 \begin{figure*}
\centering
\includegraphics[width=0.7\textwidth]{plots/PDF_125GeV_Magdipheavy_50sims_Xe_Xe3x_Xe10x_XeG3_GF_TNT.pdf}
\caption{\label{fig:125gev_Mag.dip.heavy_XeFull_TNT_GF}
Same as Fig.~\ref{fig:20gev_anapole_XeFull_TNT_GF}, but now assessing the ability of xenon experiments to break the degeneracy of the magnetic dipole (heavy mediator) and electric dipole (heavy mediator). Results are shown for a $125$ GeV dark matter and and assuming the magnetic dipole is the true model.}
\end{figure*}


In a comparison of the SI and anapole interactions, we have found that even under the most optimistic of circumstances, including time information in the analysis of Gen2 direct detection does not impact model selection. Rather, correct model identification in Gen2 experiments requires observations in multiple target elements. We have shown that for the inclusion of time information to significantly increase model selection ($\simeq \cO(10)\%$) between these two interactions for a single target element, $\simeq \cO(1000)$ ($\cO(300)$) events must be observed for a 20 (500) GeV dark matter particle. Furthermore, even if time is exploited in Gen3 xenon experiments, target complementarity must also be exploited to unequivocally differentiate these two models.

In addition to the aforementioned example, we have also provided an illustration of this analysis for a comparison of the magnetic dipole and electric dipole interactions. Quantitatively these results are quite similar to those of the SI and anapole interactions.

In the event of a putative signal, future direct detection experiments will be charged with the difficult task of illuminating the high energy behavior of dark matter solely from the observed low energy recoils. This is a particular daunting task in light of the fact that many feasible dark matter models produce nearly degenerate recoil spectra. Exploiting all of the information available, including the time information of nuclear recoils and detections in multiple experiments, will likely be required to make definitive statements regarding the true nature of dark matter.

\begin{figure*}
\centering
\includegraphics[width=0.7\textwidth]{plots/PDF_125GeV_Magdiplight_50sims_Xe_Xe3x_Xe10x_XeG3_GF_TNT.pdf}
\caption{\label{fig:125gev_Mag.dip.light_XeFull_TNT_GF}
Same as \Fig{fig:125gev_Mag.dip.heavy_XeFull_TNT_GF} but for a light mediator. }
\end{figure*}


\bigskip

\textbf{Acknowledgments.} SW is supported under the University Research Association (URA) Visiting Scholars Award Program, and by a UCLA Dissertation Year Fellowship. %Fermilab is operated by Fermi Research Alliance, LLC, under Contract No. DE-AC02-07CH11359 with the US Department of Energy. 

\appendix

\section{Model Selection Prospects in Xenon (SI Interaction)}
We present in Figs.~\ref{fig:20gev_SI_Higgs_XeFull_TNT_GF}-\ref{fig:500gev_SI_Higgs_XeFull_TNT_GF} the model selection prospects for various exposure xenon experiments, including (solid) and neglecting (dashed) information on the modulation of the rate, and assuming the SI interaction is the true model. Results are shown for $20$ GeV (\Fig{fig:20gev_SI_Higgs_XeFull_TNT_GF}), $125$ GeV (\Fig{fig:125gev_SI_Higgs_XeFull_TNT_GF}), and $500$ GeV (\Fig{fig:500gev_SI_Higgs_XeFull_TNT_GF}) dark matter. Results are similar to those presented in \Sec{sec:results} for the anapole interaction. 


\begin{figure*}
\centering
\includegraphics[width=0.7\textwidth]{plots/PDF_20GeV_SI_Higgs_50sims_Xe_Xe3x_Xe10x_XeG3_GF_TNT.pdf}
\caption{\label{fig:20gev_SI_Higgs_XeFull_TNT_GF}
Same as Fig.~\ref{fig:20gev_anapole_XeFull_TNT_GF} but for the SI interaction.}
\end{figure*}


\begin{figure*}
\centering
\includegraphics[width=0.7\textwidth]{plots/PDF_125GeV_SI_Higgs_50sims_Xe_Xe3x_Xe10x_XeG3_GF_TNT.pdf}
\caption{\label{fig:125gev_SI_Higgs_XeFull_TNT_GF}
Same as Fig.~\ref{fig:20gev_anapole_XeFull_TNT_GF} but for a $125$ GeV dark matter and the SI interaction.}
\end{figure*}


\begin{figure*}
\centering
\includegraphics[width=0.7\textwidth]{plots/PDF_500GeV_SI_Higgs_50sims_Xe_Xe3x_Xe10x_XeG3_GF_TNT.pdf}
\caption{\label{fig:500gev_SI_Higgs_XeFull_TNT_GF}
Same as Fig.~\ref{fig:20gev_anapole_XeFull_TNT_GF} but for a $500$ GeV dark matter and the SI interaction.}
\end{figure*}

\bibliographystyle{JHEP}
\bibliography{mod-sel}

\end{document}